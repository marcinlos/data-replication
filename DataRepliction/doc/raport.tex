\documentclass[11pt,pdftex,a4paper]{scrartcl}

\usepackage[utf8]{inputenc}
\usepackage{lmodern} 
\usepackage[T1]{fontenc}
\usepackage{microtype}

\usepackage[pdftex]{graphicx}
\usepackage{float}
\usepackage[hypcap]{caption}
\usepackage{subcaption}

\usepackage[pdftex]{hyperref} 
\usepackage{bookmark}

\usepackage{mathtools}
\usepackage{amssymb}
\usepackage{amsthm}
\usepackage{amsmath}

\usepackage{polski}
\usepackage[polish]{babel}
\selectlanguage{polish}

\hypersetup{
  colorlinks=true,
  urlcolor=blue,
}

\title{Stochastyczne Algorytmy Obliczeniowe}
\subtitle{ 
  Zastosowanie algorytmu genetycznego do rozwiązania problem replikacji danych
  w środowisku rozproszonym
}
\date{}
\author{
  Andrzej Kaczmarczyk
  \and
  Marcin Łoś
}

\begin{document}

\maketitle

\section{Wstęp}
Celem projektu było wybranie jednego problemu obliczeniowego, zapoznanie się z istniejącymi jego
rozwiązaniami, oraz próba ulepszenia któregoś z nich. Nasz wybór padł na problem replikacji danych
w systemach rozproszonych, oraz rozwiązanie oparte na algorytmie genetycznym, opisane w \cite{Ahmad}.

\section{Opis problemu}
Dany jest system rozproszony, składający się z \(M\) hostów \(\{H_i\}\), o pojemnościach \(s_i\), 
połączonych siecią komunikacyjną tak, że koszt przesyłu jednostki danych pomiędzy \(H_i\) i \(H_j\)
wynosi \(C(i,j)\). Istnieje także \(N\) obiektów \(\{O_i\}\), o rozmiarach \(o_i\). Host \(H_i\)
wykonuje na obiekcie \(O_k\) odpowiednio \(r^{(i)}_k\) operacji odczytu, i \(w^{(i)}_k\) operacji
zapisu. Każdy obiekt \(O_i\) znajduje się pierwotnie na jednym hoście, \(SP_i\).

Obiekty mogą zostać zreplikowane na inne hosty, tak, że suma rozmiarów obiektów zreplikowanych
na hoście \(H_i\) nie przekracza jego pojemności \(s_i\). Wszystkie hosty mają pełną wiedzę o 
replikach obiektów. Operacja odczytu obiektu \(O_k\) z hosta \(H_i\) przebiega w ten sposób, że
obiekt jest wysyłany do hosta \(H_i\) z najbliższego mu hosta zawierającego replikę \(O_k\).
Operacja zapisu natomiast odbywa się w ten sposób, że host \(H_i\) wysyła nowy stan obiektu do
hosta \(SP_k\) (pierwotnego miejsca zwierającego \(O_k\)), a ten rozsyła informację o zmianie do
pozostałych hostów zawierających repliki \(O_k\). 

Koszt sumaryczny przy danym rozłożeniu replik to suma kosztów przesyłania obiektów spowodowanego
operacjami odczytu i zapisu, przebiegającymi w opisany powyżej sposób. Koszt pojedynczego przesyłu
to iloczyn ilości przesyłanych danych (rozmiaru obiektu) i kosztu jednostkowego przesyłu między
hostami (danego przez \(C(i,j)\)). Problem polega na znalezieniu replikacji minimalizującej koszt
sumaryczny.

Nieco bardziej szczegółowy opis, wraz z wzorami na całkowity koszt znaleźć można w \cite{Ahmad}.

\section{Istniejące rozwiązania}

\section{Rozwiązanie bazowe}

\section{Wprowadzone zmiany}

\section{Podsumowanie}

\begin{thebibliography}{}

  \bibitem{Ahmad} 
  T. Loukopoulos I. Ahmad.
  \emph{Static and Adaptive Data Replication Algorithms for Fast Information Access in 
    Large Distributed Systems}

\end{thebibliography}

\end{document}