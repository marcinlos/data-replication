\documentclass[11pt,pdftex,a4paper]{scrartcl}

\usepackage[utf8]{inputenc}
\usepackage{lmodern} 
\usepackage[T1]{fontenc}
\usepackage{microtype}

\usepackage[pdftex]{graphicx}
\usepackage{float}
\usepackage[hypcap]{caption}
\usepackage{subcaption}

\usepackage[pdftex]{hyperref} 
\usepackage{bookmark}

\usepackage{mathtools}
\usepackage{amssymb}
\usepackage{amsthm}
\usepackage{amsmath}

\usepackage{polski}
\usepackage[polish]{babel}
\selectlanguage{polish}

\hypersetup{
  colorlinks=true,
  urlcolor=blue,
}

\title{Stochastyczne Algorytmy Obliczeniowe}
\subtitle{ 
  Zastosowanie algorytmu genetycznego do rozwiązania problem replikacji danych
  w środowisku rozproszonym
}
\date{}
\author{
  Andrzej Kaczmarczyk
  \and
  Marcin Łoś
}

\begin{document}

\maketitle

\section{Wstęp}

\section{Opis problemu}

\section{Istniejące rozwiązania}

\section{Rozwiązanie bazowe}

\section{Wprowadzone zmiany}

\section{Podsumowanie}

\begin{thebibliography}{}

  \bibitem{Ahmad} 
  T. Loukopoulos I. Ahmad.
  \emph{Static and Adaptive Data Replication Algorithms for Fast Information Access in 
    Large Distributed Systems}

\end{thebibliography}

\end{document}