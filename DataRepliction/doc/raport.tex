\documentclass[11pt,pdftex,a4paper]{scrartcl}

\usepackage[utf8]{inputenc}
\usepackage{lmodern} 
\usepackage[T1]{fontenc}
\usepackage{microtype}

\usepackage[pdftex]{graphicx}
\usepackage{float}
\usepackage[hypcap]{caption}
\usepackage{subcaption}

\usepackage[pdftex]{hyperref} 
\usepackage{bookmark}

\usepackage{mathtools}
\usepackage{amssymb}
\usepackage{amsthm}
\usepackage{amsmath}

\usepackage{polski}
\usepackage[polish]{babel}
\selectlanguage{polish}

\hypersetup{
  colorlinks=true,
  urlcolor=blue,
}

\title{Stochastyczne Algorytmy Obliczeniowe}
\subtitle{ 
  Zastosowanie algorytmu genetycznego do rozwiązania problem replikacji danych
  w środowisku rozproszonym
}
\date{}
\author{
  Andrzej Kaczmarczyk
  \and
  Marcin Łoś
}

\begin{document}

\maketitle

\section{Wstęp}
Celem projektu było wybranie jednego problemu obliczeniowego, zapoznanie się z istniejącymi jego
rozwiązaniami, oraz próba ulepszenia któregoś z nich. Nasz wybór padł na problem replikacji danych
w systemach rozproszonych, oraz rozwiązanie oparte na algorytmie genetycznym, opisane w \cite{Ahmad}.

\section{Opis problemu}
Dany jest system rozproszony, składający się z \(M\) hostów \(\{H_i\}\), o pojemnościach \(s_i\), 
połączonych siecią komunikacyjną tak, że koszt przesyłu jednostki danych pomiędzy \(H_i\) i \(H_j\)
wynosi \(C(i,j)\). Istnieje także \(N\) obiektów \(\{O_i\}\), o rozmiarach \(o_i\). Host \(H_i\)
wykonuje na obiekcie \(O_k\) odpowiednio \(r^{(i)}_k\) operacji odczytu, i \(w^{(i)}_k\) operacji
zapisu. Każdy obiekt \(O_i\) znajduje się pierwotnie na jednym hoście, \(SP_i\).

Obiekty mogą zostać zreplikowane na inne hosty, tak, że suma rozmiarów obiektów zreplikowanych
na hoście \(H_i\) nie przekracza jego pojemności \(s_i\). Wszystkie hosty mają pełną wiedzę o 
replikach obiektów. Operacja odczytu obiektu \(O_k\) z hosta \(H_i\) przebiega w ten sposób, że
obiekt jest wysyłany do hosta \(H_i\) z najbliższego mu hosta zawierającego replikę \(O_k\).
Operacja zapisu natomiast odbywa się w ten sposób, że host \(H_i\) wysyła nowy stan obiektu do
hosta \(SP_k\) (pierwotnego miejsca zwierającego \(O_k\)), a ten rozsyła informację o zmianie do
pozostałych hostów zawierających repliki \(O_k\). 

Koszt sumaryczny przy danym rozłożeniu replik to suma kosztów przesyłania obiektów spowodowanego
operacjami odczytu i zapisu, przebiegającymi w opisany powyżej sposób. Koszt pojedynczego przesyłu
to iloczyn ilości przesyłanych danych (rozmiaru obiektu) i kosztu jednostkowego przesyłu między
hostami (danego przez \(C(i,j)\)). Problem polega na znalezieniu replikacji minimalizującej koszt
sumaryczny.

Nieco bardziej szczegółowy opis, wraz z wzorami na całkowity koszt znaleźć można w \cite{Ahmad}.

\section{Istniejące rozwiązania}

\section{Rozwiązanie bazowe}
Jako punkt wyjściowy przyjęliśmy rozwiązanie zaproponowane w \cite{Ahmad}.

\section{Narzędzia}
Do realizacji implementacyjnych aspektów projektu wykorzystaliśmy język Python i bibliotekę
PyEvolve \cite{pyevolve}, dostarczającą różnych komponentów (np. strategie selekcji) pozwalających
budować algorytmy genetyczne. Do implementacji rozproszonej użyte zostało MPI, za pośrednictwem
Pythonowych bindingów udostępnianych przez bibliotekę mpi4py \cite{mpi4py}.

\section{Przebieg prac}
Pierwszym celem po dokładnym zapoznaniem się z artykułem, na którym bazuje projekt, było odtworzenie
zaprezentowanych w nim wyników. W tym celu zaimplementowany został dokładnie przedstawiony w nim
algorytm. Początkowo planowaliśmy użyć nieco innego generatora danych do testów, takiego, który w
naszym odczuciu mógłby lepiej odwzorowywać własności instancji problemu, które występować mogą w 
praktyce. 


\section{Podsumowanie}

\section{Możliwe kierunki rozwoju}

Jakkolwiek algorytm wyspowy nie przyniósł żadnej obserwowalnej poprawy jakości otrzymywanych rozwiązań,
w pewnym stopniu może być to spowodowane małym zróżnicowaniem populacji na poszczególnych wyspach.
Populacje wybierane są niezależnie przy użyciu tego samego schematu, i są raczej na tyle duże (80
osobników), że trudno się spodziewać, by istotnie się od siebie różniły. Niewykluczone, że stworzenie
populacji bardziej zróżnicowanych (poprzez modyfikację rozkładów, z których są losowane --tj. przez
użycie różnych sposobów ich inicjalizowania) przyniosłoby lepszy efekt.

Jednym z pomysłów, które ostatecznie nie zostały wprowadzone w życie, była relaksacja operacji
genetycznych (mutacji i krzyżowania), polegająca na tym, by wymuszanie poprawności rozwiązań niejako
przenieść do etapu ewaluacji -- dopuszczać wszystkie stworzone osobniki, jednak stosować kary przy
ewaluacji, tak, by osobnikami o największym fitnessie były osobniki z poprawnym genotypem, jednak by
zwiększyć ,,mobilność'' populacji -- umożliwić zmiany genotypu, które bez relaksacji nie byłyby 
możliwe, i tym samym być może pozwolić na odkrycie trudnych do znalezienia minimów.

Otwarta pozostaje kwestia znalezienia lepszych operatorów mutacji i krzyżowania, bardziej dostosowanych
do rozpatrywanej przestrzeni stanów. Był to jeden z pierwszych pomysłów, jednak nie osiągnęliśmy w tym
kierunku żadnych postępów. Znalezienie sensownych operatorów wewnętrznych (nie wychodzących poza --
dość nieregularną -- przestrzeń stanów) pozwoliłoby uprościć algorytm, a ze względu na pewną 
,,kompatybilność'' z przestrzenią stanów być może także polepszyć znajdowane przez ich użycie 
rozwiązania. Wydaje się to być jednak zadanie stosunkowo trudne, i trudno na chwilę obecną powiedzieć,
jakiego rodzaju operacji można by szukać.

Przemyśleć należałoby kwestię generowania danych testowych. Nasze początkowe próby przeprowadzane
były na danych generowanych inaczej, niż w artykule, jednak wróciliśmy do niego by odtworzyć wyniki
i przy nim pozostaliśmy. Nie jest on jednak w naszym odczuciu idealny. W szczególności obiekcje 
budzić może topologię połączeń -- w artykule bazowym odległości \(C(i,j)\) pomiędzy poszczególnymi
hostami są losowane z rozkładem jednostajnym ze zbioru \(\{1,\ldots,10\}\). Najbardziej oczywistym
problemem zdaje się być pogwałcenie nierówności trójkąta -- \(C\) nie jest metryką. Trudno powiedzieć,
w jaki sposób wpływa to na używany algorytm, jednak wydaje się prawdopodobnym, że mogą istnieć jakieś
zmiany/ulepszenia, dla których jest to istotne. 

\begin{thebibliography}{}

  \bibitem{Ahmad} 
  T. Loukopoulos I. Ahmad.
  \emph{Static and Adaptive Data Replication Algorithms for Fast Information Access in 
    Large Distributed Systems}

  \bibitem{pyevolve}
  Biblioteka PyEvolve,
  \url{http://pyevolve.sourceforge.net/0_6rc1/}

  \bibitem{mpi4py}
  Biblioteka mpi4py,
  \url{http://mpi4py.scipy.org/}

\end{thebibliography}

\end{document}